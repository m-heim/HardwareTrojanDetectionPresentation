\documentclass[11pt]{beamer}
\usetheme{Frankfurt}
\usecolortheme{crane}
\usepackage[utf8]{inputenc}
\usepackage[english]{babel}
\usepackage{amsmath}
\usepackage{amsfonts}
\usepackage{amssymb}
\usepackage{graphicx}
%\usepackage{circuitikz}
\author{Maximilian Heim}
\title{Hardware Trojan Detection}
%\setbeamercovered{transparent} 
%\setbeamertemplate{navigation symbols}{} 
\logo{\includegraphics[width=1.5cm]{1200px-Hsas_logo.svg.png}}
\institute{University Albstadt-Sigmaringen} 
\date{\today} 
\subject{Hardware Cyber-Security}
\begin{document}

\begin{frame}
\titlepage
\end{frame}

\begin{frame}
\tableofcontents
\end{frame}

\section{Introduction}
\subsection{Hardware Trojans}
\begin{frame}
    \frametitle{What are hardware trojans?}
    \begin{enumerate}
    \item Malicious modification of an IC
    \item Conists of a trigger and a payload
    \item Triggers: Time bombs, cominational, analog \dots
    \item Payloads: Denial of service, extraction of information, keys
    \end{enumerate}
\end{frame}
\subsection{Relevance}
\begin{frame}
    \frametitle{Why is detecting them relevant?}
    \begin{itemize}
    \item Military
    \item Finance
    \item Energy
    \item Government agencies
    \item Surveillence states
    \item Transport
    \item The list continues \ldots
    \end{itemize}

\end{frame}


\section{Detection}
\subsection{Destructive detection}
\begin{frame}
    \frametitle{Reverse Engineering}
    \begin{itemize}
        \item Removing the surface layer by layer
        \item Advantages: 
        \begin{enumerate}
            \item 100 \% detection rate
        \end{enumerate}
        \item Disadvantages:
        \begin{enumerate}
            \item Only tests a single chip
            \item Chip unusable afterwards
            \item Very time consuming
        \end{enumerate}
        \item However, 
    \end{itemize}
\end{frame}
\subsection{Nondestructive detection}
\section{Conclusion}
\begin{frame}
    \frametitle{Types of detection}
\end{frame}

\end{document}